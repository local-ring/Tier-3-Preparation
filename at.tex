\part{Algebraic Topology}

\section{Math Stackexchange Digest}

\subsection{Why we need cellular homology?} Form wikipedia:

$S^n$ admits a CW structure with two cells, one 0-cell and one n-cell. Here the n-cell is attached by the constant mapping from $S^{n-1}$ to 0-cell. Since the generators of the cellular homology groups ${H_{k}}(S^n_{k},S^{n}_{k - 1})$ can be identified with the k-cells of ``$S^n$'', we have that ${H_{k}}(S^n_{k},S^{n}_{k - 1})=\Z$ for $k = 0, n,$ and is otherwise trivial.

Hence for $n>1$, the resulting chain complex is:

\[\dotsb\overset{\partial_{n+2}}{\longrightarrow\,}0
\overset{\partial_{n+1}}{\longrightarrow\,}\Z
\overset{\partial_n}{\longrightarrow\,}0
\overset{\partial_{n-1}}{\longrightarrow\,}
\dotsb
\overset{\partial_2}{\longrightarrow\,}
0
\overset{\partial_1}{\longrightarrow\,}
\Z {\longrightarrow\,}
0,\]

but then as all the boundary maps are either to or from trivial groups, they must all be zero, meaning that the cellular homology groups are equal to
:
\[H_k(S^n) = \begin{cases} \mathbb Z & k=0, n \\ \{0\} & \text{otherwise.} \end{cases}\]

When $n=1$, it is not very difficult to verify that the boundary map $\partial_1$ is zero, meaning the above formula holds for all positive $n$.

As this example shows, computations done with cellular homology are often more efficient than those calculated by using singular homology alone.

\subsection{\href{https://math.stackexchange.com/questions/42005}{Precise official definition of a cell complex and CW-complex}}

The definition of cell complex from Introduction to Topological Manifolds (John M. Lee):

   Here are technical details of the definition. If $X$ is a non-empty topological space, a {\it cell decomposition of} $X$ is a partition $\mathcal{E}$ of $X$ into subspaces that are open cells of various dimensions, such that the following condition is satisfied:
   
   for each cell $e\in \mathcal{E}$ of dimension $n\geq 1$, there exists a continuous map $\phi$
   from some closed $n$-cell $\DD$ into $X$ (called a {\it characteristic map for $e$}) that restricts to a homeomorphism from $D^{\circ}$ (interior) onto $e$ and maps $\partial D$ into the union of all cells of $\mathcal{E}$ of dimensions strictly less than $n$.\par
   A {\it cell complex} is a Hausdorff\footnote{The Hausdorff condition is included both to rule out various pathologies (病理学) and because, as we show below, the inductive construction of cell complexes automatically yields Hausdorff spaces.} space $X$ together with a specific cell decomposition of $X$.\par
   A {\it CW complex} is cell complex $(X,\mathcal{B})$ satisfying the following additional conditions:\\
   (C) The closure of each cell is contained in a union of finitely many cells.\\
   (W) The topology of $X$ is coherent\footnote{First we need the following definition. Suppose $X$ is a topological space, and $\mathcal{B}$ is any family of subspaces of $X$ whose union is $X$. To say that the topology of $X$ is {\it coherent} with $\mathcal{B}$ means that a subset $U\subseteq X$ is open in $X$ if and only if its intersection with each $B\in \mathcal{B}$ is open in $B$.\par 
   It is easy to show by taking complements that this is equivalent to the requirement that $U$ is closed in $X$ if and only if $U\cap B$ is closed in $B$ for each $B \in\mathcal{B}$. (In either case, the ``only if'' implication always hold by definition of the subspace topology on $B$, so it is the ``if'' part that is significant. For example, if $(X_{\alpha})$ is an indexed family of topological spaces, the disjoint union topology on $\coprod X_{\alpha}$ is coherent with the family $(X_{\alpha})$, thought of as subspaces of the disjoint union.} with the family of closed subspaces $\{\bar{e}:\ e\in \mathcal{B}\}$ \par
\vspace{3pt}
\hrule
\vspace{3pt}
Let $\mathbb{B}^n$ denote a closed n-ball. As far as I know, a {\it cell complex} is a space, obtained as \[X=\bigcup_{i\in\mathbb{N}_0} X^{(i)},\] 
such that
1. $X^{(0)}$ is a discrete space and

2. $X^{(n)}$ is obtained from $X^{(n-1)}$ by attaching $n$-cells, i.e.
\[X^{(n)}= X^{(n-1)}\bigcup_{f_{\lambda}}\mathbb{B}^n= X^{(n-1)}\coprod \mathbb{B}^n/\sim\]
where $x\sim f_\lambda(x)$.

3. $A\subseteq X$ is closed in $X$ if and only if $A\cap X^{(n)}$ is closed in $X^{(n)}$ for any $n$.

But shouldn't there be some condition on $f_\lambda$? For example, if we have a graph ($1$-dimensional complex) consisting of a single vertex and a single edge. Then when we are attaching $\mathbb{B}^2$, we can set $f$ to map the whole $S^1$ to a single point on the edge, that isn't the vertex. Thus we get a very weird space:
\begin{figure}[h!]
\centering
\includegraphics[width=6cm]{hulu.png}
\end{figure}

Shouldn't $f$ go along each loop/edge in $X^{(1)}$ integer many times and not stop in the middle? Also, how do we prevent $f$ from oscillating infninitely? For example, if $X^{(1)}$ contains two edges $a,b\subseteq\{0\}\times\mathbb{R}\subseteq\mathbb{R}^2$ with $a\cap b=\{(0,0)\}$, then $f(x)=(0,x^2\sin(1/x))$ can go infinitely many times into $a$ and $b$.\hfill || \par



No part of the definition of a CW complex prevents the sorts of examples that you have indicated.  In particular:

1. It is perfectly fine for the entire boundary of a 2-cell to be attached to a single point in the middle of a 1-cell.

2. It is perfectly fine for the boundary of a 2-cell to be attached to a 1-cell in an oscillating fashion, e.g. locally resembling $x \sin(1/x)$.\par

I think you have a good understanding of the definition of a CW complex, you are just confused about how a definition this general could be useful.\par

The reason is that algebraic topology is mostly defined up to homotopy equivalence. This is usually defined as follows: two spaces $X$ and $Y$ are homotopy equivalent if there exist maps $f\colon X\to Y$ and $g\colon Y \to X$ so that $f\circ g$ and $g\circ f$ are homotopic to the identity map.  Homotopy equivalent spaces have the same homotopy groups, the same homology and cohomology groups, and essentially all the same homotopical properties.\par

The reason that wildness of the attaching maps $f_\lambda$ is unimportant is that any ``wild'' CW complex is homotopy equivalent to a ``tame'' CW complex.  In particular, the homotopy type of a CW complex is entirely determined by the homotopy classes of the attaching maps. That is, if you replace one of the attaching maps by a homotopic map, then the resulting CW complex is homotopy equivalent.\par

For example, if the entire boundary of a 2-cell is attached to the middle of a 1-cell, this is homotopic to a map that attaches the entire boundary to a nearby 0-cell, so the two resulting complexes are homotopy equivalent.  Similarly, any oscillating map like $x\sin(1/x)$ is homotopic to a more reasonable map, so any CW complex with an oscillating attachment is homotopy equivalent to one with nicer attaching maps.\par

Indeed, assuming we are willing to work up to homotopy equivalence, we can assume that the boundary of any 2-cell maps to either

1. A single 0-cell via a constant map, or

2. To a closed loop of edges in the 1-skeleton, using a map which is locally an embedding.\par

This is how topologists tend to think of cell complexes for the purposes of algebraic topology.\par

Incidentally, the reason that it is useful to allow arbitrary attaching maps is that it glosses over (掩盖,避免考虑) the problem of how to choose a ``nice'' representative for each homotopy class of maps from the boundary of an $n$-cell to the $(n-1)$-skeleton.  Though it's obvious how such ``nice'' representatives should work when $n=2$, it becomes less clear in higher dimensions.\par

For example, it is possible to have a CW complex whose $2$-skeleton is a 2-sphere, and then attach a $4$-cell to the $2$-skeleton using a non-trivial element of $\pi_3(S^2)$.  That is, the boundary of the $4$-cell is a $3$-sphere $S^3$, which is being attached to the sphere $S^2$ via the Hopf map $S^3 \to S^2$.  This is actually a very useful cell complex, because it gives a cell structure for the projective space $\mathbb{C}P^2$ with only three cells.\par

Simplicial complexes are much simpler: they are truly just combinatorial objects, with simplexes glued together using the simple linear identifications.  The disadvantage is that a simplicial complex must usually have many more simplices than the cells of a cell complex.  For example, there exists a CW complex for the torus that has only four cells, while a simplical complex homeomorphic to a torus requires at least a couple dozen simplices.  In general, putting a CW structure on a space requires only homotopy information, while putting a simplicial structure on the space requires a much more careful consideration of the geometry.\par

By the way, the best place to learn about cell complexes and their relation to homotopy equivalence is Chapter 0 of \cite{hatcher}.\hfill ||

\vspace{3pt}
\hrule
\vspace{3pt}

Another post called \href{https://math.stackexchange.com/questions/1837569}{CW complex $=$ Cell complex?} gives the difference between CW complex and cell complex:

In topological spaces, a cell complex is usually the same thing as a CW-complex. The name ``cell complex'' comes from the fact that there exists \href{https://ncatlab.org/nlab/show/cell+complex}{generalizations to other categories}), but if you're interested in topological spaces then for all intents and purposes ``cell complex'' = ``CW-complex''.\par

A finite cell complex is a cell complex that has a finite number of cells. Each cell itself is compact. If $X$ is a finite cell complex, it is a quotient space of a disjoint union of a finite number of cells, and such a disjoint union is compact, hence $X$ is compact. It can in fact be shown that a cell complex is compact iff it is finite, cf. e.g. the book of Hatcher. So in particular in your theorem, yes, the $X_i$ are compact.\par

Note however some authors take a different approach, for example there is a different definition in Lee's book, as far as I can tell, roughly speaking in a cell complex you don't have to glue cells in the order of their dimension, whereas in a CW-complex you do. The name ``CW-complex'' is never ambiguous though, and a finite cell complex (with Lee's definition) is always a CW-complex.

\subsection{\href{https://math.stackexchange.com/questions/1401483/does-trivial-fundamental-group-imply-contractible}{Does trivial fundamental group imply contractible?}}

Let $X$ be a path-connected topological space with a trivial fundamental group:
$$\pi_1(X,x_0)=\{e\}.$$

Does $X$ have to be homotopic to a point?

The converse is true: a contractible space has trivial fundamental group. But what about the converse? Does the fundamental group tells us enough of the space to fix its homotopy type when it is trivial?

No, consider the sphere $S^2$.



\section{Base space}

 \subsection{Spectral sequence}
 I need to know what is a cellular homology of a CW complex to understand the naive example of spectral sequence (See \cite{hatcher} Ch. V).\\ 
 (from Wikipedia) If $X$ is a CW-complex with n-skeleton  $X_{n}$, the cellular-homology modules are defined as the homology groups of the cellular chain complex
\[\cdots \to {H_{n+1}}(X_{n+1},X_{n})\to {H_{n}}(X_{n},X_{n-1})\to {H_{n-1}}(X_{n-1},X_{n-2})\to \cdots ,\]
where  $X_{-1}$ is taken to be the empty set.

The group ${H_{n}}(X_{n},X_{n-1})$ is free abelian, with generators that can be identified with the  $n$-cells of  $X$. Let  $e_{n}^{\alpha }$ be an $n$-cell of $X$, and let  \[ \chi _{n}^{\alpha }:\partial e_{n}^{\alpha }\cong S^{n-1}\to X_{n-1}\] be the attaching map. Then consider the composition

 \[\chi _{n}^{\alpha \beta }:S^{n-1}\,{\stackrel {\cong }{\longrightarrow }}\,\partial e_{n}^{\alpha }\,{\stackrel {\chi _{n}^{\alpha }}{\longrightarrow }}\,X_{n-1}\,{\stackrel {q}{\longrightarrow }}\,X_{n-1}/\left(X_{n-1}\setminus e_{n-1}^{\beta }\right)\,{\stackrel {\cong }{\longrightarrow }}\, S^{n-1},\]
where the first map identifies $S^{n-1}$ with $\partial e_{n}^{\alpha }$ via the characteristic map  $\Phi _{n}^{\alpha }$ of $e_{n}^{\alpha }$, the object $e_{n-1}^{\beta }$ is an  $(n-1)$-cell of $X$, the third map  $q$ is the quotient map that collapses $X_{n-1}\setminus e_{n-1}^{\beta }$ to a point (thus wrapping  $e_{n-1}^{\beta }$ into a sphere  $S^{n-1})$, and the last map identifies $X_{n-1}/\left(X_{n-1}\setminus e_{n-1}^{\beta }\right)$ with  $S^{n-1}$ via the characteristic map  $\Phi _{n-1}^{\beta }$ of  $e_{n-1}^{\beta }$.


The boundary map $d_{n}:{H_{n}}(X_{n},X_{n-1})\to {H_{n-1}}(X_{n-1},X_{n-2})$ is then given by the formula
 \[{d_{n}}(e_{n}^{\alpha })=\sum _{\beta }\deg \left(\chi _{n}^{\alpha \beta }\right)e_{n-1}^{\beta },\]
where  $\deg \left(\chi _{n}^{\alpha \beta }\right)$ is the \textit{degree}\footnotemark of  $\chi _{n}^{\alpha \beta }$ and the sum is taken over all $(n-1)$-cells of $X$, considered as generators of ${H_{n-1}}(X_{n-1},X_{n-2})$.

\footnotetext{In topology, the degree of a continuous mapping between two compact oriented manifolds of the same dimension is a number that represents the number of times that the domain manifold wraps around the range manifold under the mapping. The degree is always an integer, but may be positive or negative depending on the orientations.\par

The degree of a map was first defined by Brouwer, who showed that the degree is homotopy invariant (invariant among homotopies), and used it to prove the Brouwer fixed point theorem. In modern mathematics, the degree of a map plays an important role in topology and geometry. In physics, the degree of a continuous map (for instance a map from space to some order parameter set) is one example of a topological quantum number.\par

The simplest and most important case is the degree of a continuous map from the $n$-sphere $S^{n}$ to itself (in the case $n=1$, this is called the winding number):

Let $f\colon S^{n}\to S^{n}$ be a continuous map. Then $f$ induces a homomorphism $f_{*}\colon H_{n}\left(S^{n}\right)\to H_{n}\left(S^{n}\right)$, where $H_{n}\left(\cdot \right)$ is the $n$-th homology group. Considering the fact that $H_{n}\left(S^{n}\right)\cong {\mathbb  {Z}}$, we see that $f_{*}$ must be of the form $f_{*}\colon x\mapsto \alpha x$ for some fixed $\alpha \in {\mathbb  {Z}}$. This $\alpha$  is then called the degree of $f$.

In algebriac topology: Let $X$ and $Y$ be closed connected oriented m-dimensional manifolds. Orientability of a manifold implies that its top homology group is isomorphic to $\Z$. Choosing an orientation means choosing a generator of the top homology group.\par

A continuous map $f : X\to Y$ induces a homomorphism $f_*$ from $H_m(X)$ to $H_m(Y)$. Let $[X]$, resp. $[Y]$ be the chosen generator of $H_m(X)$, resp. $H_m(Y)$ (or the fundamental class of $X$, $Y$). Then the degree of $f$ is defined to be $f_*([X])$. In other words,
\[f_{*}([X])=\deg(f)[Y]\,.\]
If $y$ in $Y$ and $f^{-1}(y)$ is a finite set, the degree of $f$ can be computed by considering the $m$-th local homology groups of $X$ at each point in $f^{−1}(y)$.}

\section{Singular Homology and Cohomology}
feat. John Milnor

The appendix will give brief proofs of a number theorems concentrating singular cohomology theory which are needed in the text. To fix our notations and our sign conventions, we will start with basic definitions. Nevertheless we will assume some familiarity with homology and cohomology theory. In particular we will assume that the reader is acquainted with those fundamental properties which are summarized in the [Elienberg-Steenrod] axioms.

Since these lectures were first given, several texts have appeared which present cohomology theory at the level we need, notably [Hilton and Wyliel], [Spanier], and [Dold, 1972].

 \subsection{Basic definitions}
 The {\it standard $n$-simplex} is the convex set $\Delta^n\subset \R^{n+1}$
